%%%%%%%%%%%%%%%%%%%%%%%%%%%%%%%%%%%%%%%%%%%%%%%%%%%%%%%%%%%%%%%%%%%%%%%%%%%%%%%%
% Medium Length Graduate Curriculum Vitae
% LaTeX Template
% Version 1.2 (3/28/15)
%
% This template has been downloaded from:
% http://www.LaTeXTemplates.com
%
% Original author:
% Rensselaer Polytechnic Institute
% (http://www.rpi.edu/dept/arc/training/latex/resumes/)
%
% Modified by:
% Daniel L Marks <xleafr@gmail.com> 3/28/2015
%
% Important note:
% This template requires the res.cls file to be in the same directory as the
% .tex file. The res.cls file provides the resume style used for structuring the
% document.
%
%%%%%%%%%%%%%%%%%%%%%%%%%%%%%%%%%%%%%%%%%%%%%%%%%%%%%%%%%%%%%%%%%%%%%%%%%%%%%%%%

%-------------------------------------------------------------------------------
%	PACKAGES AND OTHER DOCUMENT CONFIGURATIONS
%-------------------------------------------------------------------------------

%%%%%%%%%%%%%%%%%%%%%%%%%%%%%%%%%%%%%%%%%%%%%%%%%%%%%%%%%%%%%%%%%%%%%%%%%%%%%%%%
% You can have multiple style options the legal options ones are:
%
%   centered:	the name and address are centered at the top of the page
%				(default)
%
%   line:		the name is the left with a horizontal line then the address to
%				the right
%
%   overlapped:	the section titles overlap the body text (default)
%
%   margin:		the section titles are to the left of the body text
%
%   11pt:		use 11 point fonts instead of 10 point fonts
%
%   12pt:		use 12 point fonts instead of 10 point fonts
%
%%%%%%%%%%%%%%%%%%%%%%%%%%%%%%%%%%%%%%%%%%%%%%%%%%%%%%%%%%%%%%%%%%%%%%%%%%%%%%%%

\documentclass[margin]{res}

% Default font is the helvetica postscript font
\usepackage{tgtermes}
\usepackage{aasmacros}
\usepackage{hyperref}
\usepackage{enumitem}

% Increase text height
\textheight=700pt

\setitemize{noitemsep,topsep=0pt,parsep=3pt,partopsep=0pt}
\setenumerate{noitemsep,topsep=0pt,parsep=0pt,partopsep=0pt}

\newcommand{\doi}[2]{\emph{\href{http://dx.doi.org/#1}{{#2}}}}
\newcommand{\ads}[2]{\href{http://adsabs.harvard.edu/abs/#1}{{#2}}}
\newcommand{\arxiv}[1]{{\href{http://arxiv.org/abs/#1}{arXiv:{#1}}}}

\begin{document}

%-------------------------------------------------------------------------------
%	NAME AND ADDRESS SECTION
%-------------------------------------------------------------------------------
%\name{J. Ted Mackereth}

% Note that addresses can be used for other contact information:
% -phone numbers
% -email addresses
% -linked-in profile
\address{\Large{\textbf{J. Ted Mackereth}}\\
Postdoctoral Research Fellow, \emph{University of Birmingham}}
\address{j.e.mackereth@bham.ac.uk \\ jmackereth.github.io}

% Uncomment to add a third address
%\address{Address 3 line 1\\Address 3 line 2\\Address 3 line 3}
%-------------------------------------------------------------------------------

\begin{resume}
%-------------------------------------------------------------------------------
%	EDUCATION SECTION
%-------------------------------------------------------------------------------
\section{EDUCATION \& TRAINING}
\textbf{University of Birmingham}, Birmingham, UK\\
{\sl Galactic Archaeology Research Fellow}, School of Physics \& Astronomy \hfill 2019-Present\\
\emph{ASTEROCHRONOMETRY Project (European Research Council Consolidator Grant)}

\textbf{Liverpool John Moores University}, Liverpool, UK\\
{\sl PhD}, Astrophysics Research Institute \hfill 2015 - 2019\\
\emph{Unveiling the History and Nature of the Milky Way using Galactic Surveys and Numerical Simulations}

\textbf{University of Liverpool / Liverpool John Moores University}, Liverpool, UK\\
{\sl Master of Physics (MPhys)}, Astrophysics Research Institute, 1:1\hfill 2011-2015
\\
\emph{The variation of NIR spectral lines by stellar parameters and chemical abundances}



%-------------------------------------------------------------------------------
%	PROJECTS SECTION
%-------------------------------------------------------------------------------
%-------------------------------------------------------------------------------
\section{SELECTED TALKS}
\begin{itemize}
\item[-] \emph{Linking insights into the disc, bulge and halo for a holistic approach to constraining the assembly of the Milky Way}, Contributed Talk, \emph{The Gaia Treasure Hunt}, MW-GAIA COST action workshop, Cambridge, UK
\item[-] \emph{Linking insights into the disc, bulge and halo for a holistic approach to constraining the assembly of the Milky Way}, CITA Seminar, Toronto, ON, Canada
\item[-] \emph{Constraints on the assembly of the Milky Way from APOGEE, \emph{Gaia} and the EAGLE simulations}, Invited Talk, KITP Conference: \emph{`In the balance: stasis and disequilibrium in the Milky Way'}, Santa Barbara, CA, USA
\item[-] \emph{Constraining the formation of the Milky Way disk with APOGEE, Gaia and the EAGLE simulations}, Invited Plenary Talk, SDSS-IV Collaboration Meeting 2018, Seoul, South Korea
\item[-]\emph{The origin of diverse $\alpha$-element enrichment in galaxy discs}, Invited Lunch Seminar, ICC, Durham University
\item[-] \emph{Constraints on the origin of the high-$\mathrm{[\alpha/Fe]}$ disc with APOGEE-\emph{Gaia},} Contributed Talk, \emph{Gaia: The billion-star galaxy census: at the threshold of Gaia data release 2}, EWASS 2018, Liverpool, UK
\item[-] \emph{Contextualising $\mathrm{[\alpha/Fe]}$ bimodality in the EAGLE simulations}, Contributed Talk \emph{Hello, goodbye: understanding the duality of the Milky Way}, EWASS 2018, Liverpool, UK
\item[-] \emph{Galactic Archaeology with mono-age stellar populations}, Invited Seminar, March 2018, University of Birmingham
\item[-] \emph{The Milky Way in a cosmological context: The origin of diverse $\alpha$-element enrichment in galaxy discs}, Contributed Talk, Virgo Collaboration Meeting 2017, Garching, Germany
\item[-] \emph{The origin of $\alpha$-element bimodality in the Milky Way and galaxy discs}, Informal Talk, Flatiron CCA Stars Group Meeting, New York City, USA
\item[-] \emph{Reconstructing the history of the Milky Way disc}, Poster Prize Talk, IAUS334: \emph{Rediscovering our Galaxy}, Potsdam, Germany
\item[-] \emph{Reconstructing the history of the Milky Way disc}, Contributed Talk, \emph{Bridging the near and the far: from the Milky Way to nearby galaxies}, EWASS 2017, Prague, Czech Republic
\item[-] \emph{Constraining models for Galactic disk formation with APOGEE and EAGLE}, Contributed Talk, SDSS-IV Collaboration Meeting 2016, Madison, WI
\end{itemize}

\newpage

\section{WORKSHOPS}
\begin{itemize}
\item[-] \emph{Weighing Stars from Birth to Death: How to Determine Stellar Masses?}, 2018 Lorentz Center Workshop (Invited), Leiden, The Netherlands
\item[-] \emph{2018 \emph{Gaia} Sprint}, CCA, Flatiron Institute, New York City, USA
\item[-] \emph{2017 \emph{Gaia} Sprint}, \emph{MPIA}, Heidelberg, Germany
\end{itemize}

\section{GRANTS, AWARDS \& HONOURS}
\begin{itemize}
\item[-] LJMU Faculty of Engineering Thesis prize, 2019, 100 GBP
\item[-] SDSS Early Career Travel Fund Grant, 2018, 600 USD
\item[-] Dunlap Institute Visiting Member, University of Toronto, Canada, 2017, 1800 CAD
\item[-] RAS Personal Grant, \emph{The kinematics and dynamics of mono-abundance populations in the Milky Way using Gaia and APOGEE}, 2017, 1000 GBP
\item[-] IAUS334 Travel Grant, 2017, 300 EUR
\item[-] Poster Prize, IAUS334, Potsdam, Germany, 2017
\end{itemize}

\section{OTHER AFFILIATIONS}
\begin{itemize}
\item[-] SDSS `Milky Way as a Galaxy' Working Group Co-Chair
\item[-] SDSSIV/APOGEE-2 Team member
\item[-] APO-K2 Core science team member
\item[-] WEAVE survey Galactic Archaeology science working group member
\item[-] Maunkea Spectroscopic Explorer science working group member
\item[-] Virgo Consortium member
\item[-] \textbf{Reviewer:} MNRAS, ApJ, A\&A, CanTAC
\end{itemize}

\section{SOFTWARE}
\begin{itemize}
\item[-] \emph{galpy} galactic dynamics package contributor
\item[-] \emph{apogee} python package contributor
\item[-] \textbf{Languages:} Python, R, Stan, Tensorflow, SQL/ADQL, \LaTeX, HTML/CSS.
\end{itemize}

\section{STUDENT MENTORING}
Danny Horta-Darrington, \emph{PhD Student}, LJMU, 2018-Present\\
Emma Willett, \emph{PhD Student}, UoB, 2019-Present\\
Rayhan Mahmud, \emph{Undergraduate Summer Student}, UoB, 2019

\section{TEACHING}
\textbf{University of Birmingham}\\
- Demonstrator, \emph{Introduction to Computing}\\
- Instructor, \emph{Computational Physics}\\
- Instructor, \emph{Postgraduate Research Skills}\\
\textbf{Liverpool John Moores University}\\
- Senior Demonstrator, \emph{Practical Astrophysics}\\
- Tutor, \emph{Computational Galactic Dynamics, Distance Learning MSc}\\
- Teaching Assistant, \emph{Introduction to Astrophysics}\\
- Mentoring of \emph{MPhys} project students\\
- PhD student talks organiser, 2017

\section{OUTREACH ACTIVITIES}
\par
- Featured on Japanese national broadcaster NHK's \emph{Cosmic Front NEXT} documentary, \emph{`Miracle Milky Way'}\\
- Consulted by the BBC for scientific input on upcoming documentary series, \emph{The Universe}\\
- Tim Peake Cosmic Classroom Event, February 2016 \emph{World Museum, Liverpool}\\
- The Size of the Universe (talk/workshop), April 2016 \emph{Werneth Primary School, Oldham, UK}\\
- Travelling to Space (talk/workshop), March 2018 \emph{Abraham Moss Community School, Manchester, UK}\\
- Undergraduate open days student representative, 2016-2018 \emph{Liverpool John Moores University / University of Liverpool}

%-------------------------------------------------------------------------------

%-------------------------------------------------------------------------------
%	EXPERIENCE SECTION
%-------------------------------------------------------------------------------
% Modify the format of each position
\begin{format}
\title{l}\employer{r}\\
\dates{l}\location{r}\\
\body\\
\end{format}

\end{resume}
\end{document}
