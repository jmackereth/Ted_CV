%%%%%%%%%%%%%%%%%%%%%%%%%%%%%%%%%%%%%%%%%%%%%%%%%%%%%%%%%%%%%%%%%%%%%%%%%%%%%%%%
% Medium Length Graduate Curriculum Vitae
% LaTeX Template
% Version 1.2 (3/28/15)
%
% This template has been downloaded from:
% http://www.LaTeXTemplates.com
%
% Original author:
% Rensselaer Polytechnic Institute 
% (http://www.rpi.edu/dept/arc/training/latex/resumes/)
%
% Modified by:
% Daniel L Marks <xleafr@gmail.com> 3/28/2015
%
% Important note:
% This template requires the res.cls file to be in the same directory as the
% .tex file. The res.cls file provides the resume style used for structuring the
% document.
%
%%%%%%%%%%%%%%%%%%%%%%%%%%%%%%%%%%%%%%%%%%%%%%%%%%%%%%%%%%%%%%%%%%%%%%%%%%%%%%%%

%-------------------------------------------------------------------------------
%	PACKAGES AND OTHER DOCUMENT CONFIGURATIONS
%-------------------------------------------------------------------------------

%%%%%%%%%%%%%%%%%%%%%%%%%%%%%%%%%%%%%%%%%%%%%%%%%%%%%%%%%%%%%%%%%%%%%%%%%%%%%%%%
% You can have multiple style options the legal options ones are:
%
%   centered:	the name and address are centered at the top of the page 
%				(default)
%
%   line:		the name is the left with a horizontal line then the address to
%				the right
%
%   overlapped:	the section titles overlap the body text (default)
%
%   margin:		the section titles are to the left of the body text
%		
%   11pt:		use 11 point fonts instead of 10 point fonts
%
%   12pt:		use 12 point fonts instead of 10 point fonts
%
%%%%%%%%%%%%%%%%%%%%%%%%%%%%%%%%%%%%%%%%%%%%%%%%%%%%%%%%%%%%%%%%%%%%%%%%%%%%%%%%

\documentclass[margin]{res}  

% Default font is the helvetica postscript font
\usepackage{helvet}

% Increase text height
\textheight=700pt

\begin{document}

%-------------------------------------------------------------------------------
%	NAME AND ADDRESS SECTION
%-------------------------------------------------------------------------------
%\name{J. Ted Mackereth}

% Note that addresses can be used for other contact information:
% -phone numbers
% -email addresses
% -linked-in profile
\address{\Large{\textbf{J. Ted Mackereth}}\\
Postdoctoral Research Fellow, \emph{University of Birmingham}}
\address{j.e.mackereth@bham.ac.uk \\ jmackereth.github.io}


% Uncomment to add a third address
%\address{Address 3 line 1\\Address 3 line 2\\Address 3 line 3}
%-------------------------------------------------------------------------------

\begin{resume}


%-------------------------------------------------------------------------------
%	EDUCATION SECTION
%-------------------------------------------------------------------------------
\section{EDUCATION \& TRAINING}
\textbf{University of Birmingham}, Birmingham, UK\\
{\sl Galactic Archaeology Research Fellow}, School of Physics \& Astronomy \hfill 2019-Present\\
\emph{ASTEROCHRONOMETRY Project (European Research Council Consolidator Grant)}

\textbf{Liverpool John Moores University}, Liverpool, UK\\
{\sl PhD}, Astrophysics Research Institute \hfill 2015 - 2019\\
\emph{Unveiling the History and Nature of the Milky Way using Galactic Surveys and Numerical Simulations}

\textbf{University of Liverpool / Liverpool John Moores University}, Liverpool, UK\\
{\sl Master of Physics (MPhys)}, Astrophysics Research Institute, 1:1\hfill 2011-2015
\\
\emph{The variation of NIR spectral lines by stellar parameters and chemical abundances}

%-------------------------------------------------------------------------------

%-------------------------------------------------------------------------------
%	PROJECTS SECTION
%-------------------------------------------------------------------------------
\section{PAPERS\\ As first author}
\par
%- \textbf{Mackereth JT}, Bovy J et al. \emph{Weighing the stellar constituents of the Galactic halo with APOGEE red giant stars}, 2019, MNRAS, submitted. (arXiv:XXXX.XXXX) \\
- \textbf{Mackereth JT}, Bovy J, Leung HW et al. \emph{Dynamical heating across the Milky Way disc using APOGEE and Gaia}, 2019, MNRAS, 489(1) (arXiv:1901.04502) \\
- \textbf{Mackereth JT}, Schiavon RP, Pfeffer J et al. \emph{The origin of accreted stellar halo populations in the Milky Way using APOGEE, Gaia, and the EAGLE simulations}, 2019, MNRAS, 482(3) (arXiv: 1808.00968) \\
- \textbf{Mackereth JT} and Bovy J \emph{Fast estimation of orbital parameters in Milky-Way-like potentials}, 2018, PASP, 130:993 (arXiv:1802.02592)
\newline - \textbf{Mackereth JT}, Crain RA, Schiavon RP et al., \emph{The origin of diverse $\alpha$-element enrichment in galaxy discs}, 2018, MNRAS, 477(4) (arXiv: 1801.03593) 
\newline - \textbf{Mackereth JT}, Bovy J, Schiavon RP et al. \emph{The age-metallicity structure of the Milky Way disc using APOGEE}, 2017, MNRAS, 471(3) (arXiv: 1706.00018)

\section{As Co-Author}
- Trick WH, Fragkoudi F, Hunt JAS, \textbf{Mackereth JT} et al. \emph{Identifying resonances of the Galactic bar in Gaia DR2: Clues from action space} 2019, MNRAS, Submitted. 
\newline - Bovy J, Leung HW, Hunt JAS, \textbf{Mackereth JT} et al. \emph{Life in the fast lane: a direct view of the dynamics, formation, and evolution of the Milky Way's bar} 2019, MNRAS, Submitted. 
\newline - Vincenzo, F, Miglio, A, Kobayashi, C, \textbf{Mackereth, JT} et al. \emph{He abundances in disc galaxies -- I. Predictions from cosmological chemodynamical simulations} 2019, A\&A, in press. 
\newline - Hunt JAS, Bub MW, Bovy J et al. (incl. \textbf{JTM}) \emph{Signatures of resonance and phase mixing in the Galactic disk} 2019, MNRAS, in press.
\newline - Aguado DS, Ahumada R, Almeida A et al. (incl. \textbf{JTM}) \emph{The Fifteenth Data Release of the Sloan Digital Sky Surveys: First Release of MaNGA-derived Quantities, Data Visualization Tools, and Stellar Library} 2019, ApJS, 240(23)
\newline - Boecker A, Leaman R, van de Ven G et al. (incl. \textbf{JTM}) \emph{A galaxy's accretion history unveiled from its integrated spectrum} 2019, MNRAS, Submitted.
\newline - Abolfathi B, Aguado DS, Aguilar G et al. (incl. \textbf{JTM}) \emph{The Fourteenth Data Release of the Sloan Digital Sky Survey: First Spectroscopic Data from the Extended Baryon Oscillation Spectroscopic Survey and from the Second Phase of the Apache Point Observatory Galactic Evolution Experiment} 2018, ApJS, 235(2)
\newline - Albareti, FD, Allende Prieto C, Almeida A et al. (incl. \textbf{JTM}) \emph{The 13th Data Release of the Sloan Digital Sky Survey: First Spectroscopic Data from the SDSS-IV Survey Mapping Nearby Galaxies at Apache Point Observatory} 2017, ApJS, 233(2)
\newline - Schiavon RP, Zamora O, Carrera R et al. (incl. \textbf{JTM}) \emph{Chemical tagging with APOGEE: discovery of a large population of N-rich stars in the inner Galaxy} 2017, MNRAS, 465(1)

\section{Conference Proceedings}
- \textbf{Mackereth JT}, Bovy J, Schiavon RP and SDSS-IV/APOGEE Collaboration \emph{The age-metallicity structure of the Milky Way disc with APOGEE} Rediscovering our Galaxy, IAU Symposium Vol. 334
%-------------------------------------------------------------------------------
\section{TALKS \& WORKSHOPS}
- Linking insights into the disc, bulge and halo for a holistic approach to constraining the assembly of the Milky Way, \emph{Contributed Talk, 'The Gaia Treasure Hunt', MWGAIA COST action workshop}, Cambridge, UK\\
- Linking insights into the disc, bulge and halo for a holistic approach to constraining the assembly of the Milky Way, \emph{CITA Seminar}, Toronto, ON, Canada\\ 
- Constraints on the assembly of the Milky Way from APOGEE, \emph{Gaia} and the EAGLE simulations, \emph{Invited Talk, KITP Conference: `In the balance: stasis and disequilibrium in the Milky Way'}, Santa Barbara, CA, USA\\ 
- Introduction to \emph{Gaia} data using ADQL, \emph{Postgraduate Skills Session, University of Birmingham}, Birmingham, UK\\
- Weighing Stars from Birth to Death: How to Determine Stellar Masses?, \emph{2018 Lorentz Center Workshop (Invited)}, Leiden, The Netherlands\\
- Constraining the formation of the Milky Way disk with APOGEE, Gaia and the EAGLE simulations, \emph{Invited Plenary Talk, SDSS-IV Collaboration Meeting 2018}, Seoul, South Korea\\
- 2018 \emph{Gaia} Sprint Participant, \emph{CCA, Flatiron Institute}, New York City, USA\\
- The origin of diverse $\alpha$-element enrichment in galaxy discs, \emph{Friday Lunch Astronomy Talk}, ICC, Durham University\\
- Constraints on the origin of the high-$\mathrm{[\alpha/Fe]}$ disc with APOGEE-\emph{Gaia}, \emph{Gaia: The billion-star galaxy census: at the threshold of Gaia data release 2, EWASS 2018}, Liverpool, UK \\
- Contextualising $\mathrm{[\alpha/Fe]}$ bimodality in the EAGLE simulations, \emph{Hello, goodbye: understanding the duality of the Milky Way, EWASS 2018}, Liverpool, UK \\
- Galactic Archaeology with mono-age stellar populations, \emph{BISON Group talk}, March 2018, University of Birmingham\\
- The Milky Way in a cosmological context: The origin of diverse $\alpha$-element enrichment in galaxy discs, \emph{Virgo Collaboration Meeting 2017}, Garching, Germany\\
- The origin of $\alpha$-element bimodality in the Milky Way and galaxy discs, \emph{Informal Talk, Flatiron CCA Stars Group Meeting}, New York City, USA\\
- Dunlap Institute Visiting Member, \emph{Dunlap Institute, University of Toronto}, Toronto ON, Canada\\
- 2017 Gaia Sprint Participant, \emph{MPIA}, Heidelberg, Germany\\
- Reconstructing the history of the Milky Way disc, \emph{Poster Prize Talk, IAUS334: Rediscovering our Galaxy}, Potsdam, Germany\\
- Reconstructing the history of the Milky Way disc, \emph{Bridging the near and the far: from the Milky Way to nearby galaxies, EWASS 2017}, Prague, Czech Republic\\
- Constraining models for Galactic disk formation with APOGEE and EAGLE, \emph{SDSS-IV Collaboration Meeting 2016}, Madison, WI\\



\section{GRANTS \& AWARDS}
\par
- 2019 LJMU Faculty of Engineering Best Thesis prize \\
- 2018 SDSS Early Career Travel Fund Grant, \emph{USD 600}\\
-  2017 Dunlap Visitor Grant, \emph{CAD 1800} Dunlap Institute, University of Toronto\\
- 2017 RAS Personal Grant, \emph{GBP 1000} The kinematics and dynamics of mono-abundance populations in the Milky Way using Gaia and APOGEE\\
- IAUS334 Travel Grant, \emph{EUR 280}\\
- Poster Prize, \emph{IAUS334}, Potsdam, Germany\\



\section{OTHER AFFILIATIONS}
\par
- SDSS `Milky Way as a Galaxy' Working Group Co-Chair \\
- SDSSIV/APOGEE-2 Team member\\
- APO-K2 Core science team member\\
- WEAVE survey Galactic Archaeology science working group Member\\
- \textbf{Reviewer:} MNRAS, ApJ, A\&A, CanTAC\\


\section{SOFTWARE}
\par
- Developer of \emph{sewingmachine} equivalent width code\\
- \emph{galpy} galactic dynamics package contributor\\
- \emph{apogee} python package contributor\\
\textbf{Languages:} Python, R, Stan, Tensorflow, SQL/ADQL, \LaTeX.\\

\section{TEACHING \& MENTORING}
\par 
\textbf{University of Birmingham}\\
- Demonstrator, \emph{Computational Physics}\\
\textbf{Liverpool John Moores University}\\
- Senior Demonstrator, \emph{Practical Astrophysics}\\
- Tutor, \emph{Computational Galactic Dynamics, Distance Learning MSc}\\
- Teaching Assistant, \emph{Introduction to Astrophysics}\\
- Mentoring of \emph{MPhys} project students\\
- PhD student talks organiser, 2017

\section{OUTREACH ACTIVITIES}
\par
- Featured on Japanese national broadcaster NHK's \emph{Cosmic Front NEXT} programme on the `Miracle Milky Way'\\
- Tim Peake Cosmic Classroom Event, February 2016 \emph{World Museum, Liverpool}\\
- The Size of the Universe (talk/workshop), April 2016 \emph{Werneth Primary School, Oldham, UK}\\
- Travelling to Space (talk/workshop), March 2018 \emph{Abraham Moss Community School, Manchester, UK}\\
- Undergraduate open days student representative, 2016-2018 \emph{Liverpool John Moores University / University of Liverpool}

%-------------------------------------------------------------------------------
%	COMPUTER SKILLS SECTION
%-------------------------------------------------------------------------------
\section{SPECIFIC SKILLS}
- Stellar spectroscopy\\
- Numerical simulations of galaxy formation\\
- Statistical modelling and inference in multidimensional data\\
- Analysis of massive stellar surveys (APOGEE, \emph{Gaia}, WEAVE) \\
\\

%-------------------------------------------------------------------------------

%-------------------------------------------------------------------------------
%	EXPERIENCE SECTION
%-------------------------------------------------------------------------------
% Modify the format of each position
\begin{format}
\title{l}\employer{r}\\
\dates{l}\location{r}\\
\body\\
\end{format}
%-------------------------------------------------------------------------------

%-------------------------------------------------------------------------------
%	Interests
%-------------------------------------------------------------------------------
%\section{INTERESTS}

%-------------------------------------------------------------------------------
\end{resume}
\end{document}